Inicio
    // Declarar la variable para almacenar el número //
    
    Definir numero Como Entero

    // Solicitar al usuario que ingrese un número entero//
    
    Escribir "Ingrese un número entero:"
    
    Leer numero

    // Verificar si el número es par o impar//
    
    Si (numero % 2 = 0) Entonces
        // Si el residuo de la división por 2 es 0, el número es par//
        
        Escribir "El número ", numero, " es par."
        
    Sino
    
        // Si el residuo de la división por 2 no es 0, el número es impar//
        
        Escribir "El número ", numero, " es impar."
        
    Fin Si
    
Fin


//
HTML (HyperText Markup Language) es un lenguaje utilizado para crear la estructura y el contenido de una página web  "maquetado". Permite definir elementos como encabezados, párrafos, listas, enlaces, imágenes y otros componentes que forman la base de un documento web.

CSS (Cascading Style Sheets) es un lenguaje de estilo que se utiliza para dar presentación y el diseño de un documento HTML. Permite aplicar estilos, como colores, fuentes, márgenes y disposición de los elementos en la página, mejorando así la apariencia visual del contenido.

Diferencias:

Propósito: HTML se utiliza para definir la estructura y el contenido de una página web, mientras que CSS se utiliza para controlar su presentación visual.
Sintaxis: HTML utiliza etiquetas (por ejemplo, <h1>, <p>) para marcar el contenido, mientras que CSS utiliza selectores y propiedades (por ejemplo, h1 { color: red; }) para aplicar estilos.
Interacción: HTML es esencial para la creación de una página web, mientras que CSS es opcional, pero se recomienda para mejorar la estética. Sin CSS, una página HTML puede ser funcional, pero no atractiva.

2. Describe la diferencia entre una clase y un ID en CSS.

Class: Se define con un punto (.) seguido del nombre de la clase (por ejemplo, .mi-clase).
Puede ser utilizada en múltiples elementos dentro de una página HTML, permitiendo aplicar el mismo estilo a varios elementos.
Es útil para agrupar estilos que se aplican a varios elementos similares.

ID: Se define con un símbolo de numeral (#) seguido del nombre del ID (por ejemplo, #mi-id).
Debe ser único en una página HTML, lo que significa que solo puede ser utilizado en un elemento específico.
Es útil para estilos que necesitan ser aplicados a un único elemento o para seleccionar un elemento específico para manipulación con JavaScript.

3. Menciona al menos tres ventajas de separar el contenido HTML de los estilos CSS.

Mantenibilidad:
Separar HTML y CSS facilita la actualización y el mantenimiento del sitio web. Los cambios en el estilo se pueden realizar en un solo archivo CSS sin necesidad de modificar el HTML, lo que reduce la posibilidad de errores.

Reusabilidad:
Los estilos CSS pueden ser aplicados a múltiples elementos HTML en diferentes páginas, lo que permite mantener una apariencia consistente y reduce la duplicación de código.

Rendimiento:
Al separar los estilos en un archivo CSS, los navegadores pueden almacenar en caché el archivo CSS. Esto mejora el rendimiento de la carga de la página, ya que el navegador no necesita volver a cargar los estilos cada vez que se accede a una nueva página del sitio.
//
